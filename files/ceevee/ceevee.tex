\documentclass[11pt,letter]{moderncv}
\moderncvtheme[purple]{classic}
\usepackage[margin=0.75in]{geometry}
\usepackage{fancyhdr}
\usepackage{etoolbox}
\usepackage{lastpage}

\pagestyle{fancy}
\fancyhf{}
 
\rfoot{\addressfont\color{color2}\footnotesize{Page \thepage \hspace{1pt} of \pageref{LastPage}}}

\makeatletter
\patchcmd{\makecvtitle}{\ifthenelse{\equal{\@addresscountry}{}}}%
{\ifstrempty{\@addresscountry}}{}{}
\makeatother

\renewcommand{\footrulewidth}{0.5pt}
\cfoot{\addressfont\color{color2}\footnotesize Curriculum Vitae}
\lfoot{\addressfont\color{color2}\footnotesize Edgar Jaramillo Rodriguez}
\usepackage[usestackEOL]{stackengine}
\setstackgap{L}{12pt}

\firstname{Edgar}
\familyname{Jaramillo Rodriguez}
\title{Curriculum Vitae}
\address{University of California, Davis}{Department of Mathematics}
\phone{714.325.4500}
\email{ejaramillo@ucdavis.edu}
%\extrainfo{Active DOE "Q" and SCI Clearances
%}
\extrainfo{https://ejaramillorod.github.io/}

\begin{document}
\makecvtitle
{
%%%%%%%%%%%%%%%%%%
%%%%%% EDUCATION
%%%%%%%%%%%%%%%%%%
\section{Education}
\vspace{0.05in}
\cventry{In progress}{Ph.D., Mathematics}{University of California, Davis}{Advisor: Dr. Jesus De Loera}{}{}
\vspace{0.05in}
\cventry{2021}{M.A., Mathematics}{University of California, Davis}{}{}{}
\vspace{0.05in}
\cventry{2018}{B.A., Applied Mathematics}{University of California, Berkeley}{}{}{}

%%%%%%%%%%%%%%%%%%
%%%%%%  PUBLICATIONS
%%%%%%%%%%%%%%%%%%


\section{Publications}
\vspace{0.05in}
% Journal Articles
\cvitem{Articles}{{[1] J. De Loera, \textbf{ E. Jaramillo Rodriguez}, D. Oliveros, and A. Torres Hernadez, ``A Model for Birdwatching and other Chronological Sampling Activities,'' arXiv:2205.05743  (2022). Scheduled to appear in \emph{The American Mathematical Monthly.}}}
\vspace{0.05in}
\cvitem{}{[2] N. Benjamin, G. Fickes, E. Fiorini, \textbf{ E. Jaramillo Rodriguez}, E. Jovinelly, and T.W.H. Wong, ``Primes and Perfect Powers in the Catalan Triangle,'' Journal of Integer Sequences, Vol. 22 (2019), Issue 7, Paper 6}}
\vspace{0.05in}
\cvitem{}{[3] S. D. Han, \textbf{E. Jaramillo Rodriguez} and J. Yu, ``SEAR: A Polynomial- Time Multi-Robot Path Planning Algorithm with Expected Constant-Factor Optimality Guarantee,'' 2018 IEEE/RSJ International Conference on Intelligent Robots and Systems (2018), pp. 1-9.}
\vspace{0.05in}

% Pre-Prints
\cvitem{Pre-Prints}{{[1] \textbf{ E. Jaramillo Rodriguez} ``Combinatorial Methods for Barcode Analysis,'' arXiv:2206.05613 (2022).  Under review.}}
\vspace{0.05in}
% Conference Papers


%%%%%%%%%%%%%%%%%%
%%%%%%  Presentations
%%%%%%%%%%%%%%%%%%

\section{Presentations}
\vspace{0.05in}
\cvitem{2022}
{\textbf{Applied Topology in Albany Seminar}, University of Albany (remote), October 2022.  \emph{Combinatorial Methods for Barcode Analysis}.}
\vspace{0.05in}
\cvitem{}
{Society for Industrial and Applied Math Conference on Discrete Mathematics 2022 \textbf{(SIAM-DM22)}, Carnegie Mellon University, June 2022.  \emph{Barcode Posets: Combinatorial Properties and Connections}.}
\vspace{0.05in}
\cvitem{}
{\textbf{Algebraic Statistics 2022}, University of Hawai'i at Manoa, May 2022.  \emph{A Model for Birdwatching and other Chronological Sampling Activities}.}
\vspace{0.05in}
\cvitem{2021}
{ \textbf{AIM Latinx Mathematician Research Community Meeting} (virtual) December 2021.  \emph{New Methods for Mixture Model Analysis} with A. Chavez, J. De Loera,  J. Simental Rodriguez, and A. Torres Hernandez. }
\vspace{0.05in}
\cvitem{}{\textbf{UC Davis Graduate Student Research Seminar}, University of California, Davis, April 2021. \emph{Coupon Collecting, Bird Watching, and other Chronological Sampling Activities: A Random Interval Graph Model}.}
\vspace{0.05in}
\cvitem{2016}{ \textbf{Young Mathematicians Conference 2016}, The Ohio State University, August 2016. \emph{Unique Integers on the Catalan Triangle} (poster presentation) .}
\vspace{0.05in}

%%%%%%%%%%%%%%%%%%
%%%%%% PROFESSIONAL EXPERIENCE
%%%%%%%%%%%%%%%%%%
\newpage
\section{Professional Experience}
\vspace{0.05in}
\cventry{2022}{Graduate Student Intern in the Information Systems and Modeling Group (A-1)}{Los Alamos National Lab}{Mentor: Dr. Sara Del Valle}{}{}{}
\vspace{0.05in}
\cventry{2020}{NSF Mathematical Sciences Graduate Research Internship}{Lawrence Berkeley National Lab}{Mentor: Dr. Nicole Sanderson}{}{}{}
\vspace{0.05in}
\cventry{2017}{Research Experience for Undergraduates}{Center for Discrete Math and Theoretical Computer Science at Rutgers University}{Mentor: Dr.  Jingjin Yu}{}{}{}
\vspace{0.05in}
\cventry{2016}{Research Experience for Undergraduates}{Muhlenberg College}{Mentor: Dr. Eugene Fiorini}{}{}

\section{Fellowships, Awards, and Other Funding}
\vspace{0.05in}
\cvitem{2022}{Yueh-Jing Lin Fund Award Recepient, UC Davis.}
\vspace{0.05in}
\cvitem{2021-2022}{NSF Alliances for Graduate Education and the Professoriate Graduate Research Supplement (AGEP-GRS), Project title: Two-Way Research Street: Geometric Algorithms in Optimization and Computer-Based Discrete Geometry.}
\vspace{0.05in}
\cvitem{2018-2020}{Eugene Cota Robles Graduate Fellowship, UC Davis }

\section{Teaching Experience}
\vspace{0.05in}
\cvitem{Teaching Assistant}
{\textbf{Math 108: Introduction to Abstract Math}, Fall 2022, UC Davis. Lead Instructor: Dr. Orsola Capovilla-Searle.}
\cvitem{}
{\textbf{Math 21B: Integral Calculus}, Spring 2021, UC Davis. Lead Instructor: Dr. Ben Morris.}
\vspace{0.05in}
\cvitem{}{\textbf{Math 22A: Linear Algebra}, Winter 2021, UC Davis. Lead Instructor: Dr. David Marsico.}
\vspace{0.05in}
\cvitem{}{\textbf{Math 21A: Differential Calculus}, Fall 2020, UC Davis. Lead Instructor: Dr. Daniel Martin.}
\vspace{0.05in}
\cvitem{}{\textbf{Math 54: Linear Algebra}, Fall 2017, UC Berkeley. Lead Instructor: Dr. Michael Hutchings.}
\vspace{0.05in}
\cvitem{Classroom Aid}{\textbf{Martin Luther King Jr Middle School}, Fall 2015, Berkeley, CA. Lead Instructor: Jay Cohen.}

\section{Mentoring Experience}
\vspace{0.05in}
\cventry{2021}{Directed Reading Program Mentor}{UC Davis}{}{}{}{}
\begin{itemize}
\cventry{}{\item DRP Mentor} {Led two undergraduate students in a directed reading program on mathematical optimization.  I was responsible for creating the syllabus, compiling reading materials, producing exercises to guide readings, and facilitating discussions. Topics covered: Linear Regression/ Ordinary Least Squares, Ridge Regression, LASSO regression, Singular Value Decomposition/ Principle Component Analysis, and implementations of these methods in Python for real life data. In 2022 we are going to study recommender systems}{}{}{}{}
\end{itemize}
\vspace{0.05in}
\cventry{2015-2017}{Bridging Berkeley Program Director}{UC Berkeley Public Service Center}{}{}{}{}
\begin{itemize}
\cventry{}{\item Program Director} {Bridging Berkeley is a math mentoring program that matches UC Berkeley work-study students and volunteers with Berkeley middle school youth, especially those who will be first-generation college students. As director, I oversaw a team of roughly 30 mentors and 100 mentees while also mentoring a number of students myself}{}{}{}{}
\end{itemize}
\vspace{0.05in}


\section{Service}
\cvitem{Reviewer}{Boletin de la Sociedad Matemática Mexicana.}

\section{Workshops, Summer Schools, and Similar Activities}
\cvitem{2021}{Algorithmic Advances for Statistical Inference with Combinatorial Structure Workshop, Simons Insitute for the Theory of Computing (Virtual). October 11-15, 2021 }
\vspace{0.05in}
\cvitem{}{VII Mexican Workshop in Geometric and Topological Data Analysis, Centro de Investigación en Matemáticas (Virtual). September 22-29, 2021}
\vspace{0.05in}
\cvitem{}{School on Modern Directions in Discrete Optimization, Hausdorff Center for Mathematics (Virtual). September 13-17, 2021 }
\vspace{0.05in}
\cvitem{}{Geometric Methods in Optimization and Sampling Boot Camp, Simons Insitute for the Theory of Computing (Virtual). August 30-September 3, 2021 }
\vspace{0.05in}
\cvitem{}{American Institute of Mathematics (AIM) Latinx Mathematicians Research Community Meeting (Virtual). June 7-11, 2021 }
\vspace{0.05in}
\cvitem{2019}{Mixed Integer Programming Workshop,  Massachusetts Institute of Technology. July 15-18, 2019.}

\section{Skills}
\cvitem{Computer Skills}{Expertise in Python, SQL, Git, Microsoft Office, LaTeX. Proficiency in Sage, Matlab.}
\vspace{0.05in}
\cvitem{Languages}{Fluency: English and Spanish. Proficieny: French.}
\vspace{0.05in}

\end{document}
